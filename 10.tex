\subsection{100}
\begin{myfrag}
Beschreibe das Gittergasmodell. Was ist der Zusammenhang mit dem Ising
Modell?
\end{myfrag} 
\subsection{101}
\begin{myfrag}
Schätze die Freie Energie einer Domänwand im Ising Modell in ein und zwei
Dimensionen ab und diskutiere die Möglichkeit eines Phasenüberganges.
Was versteht man unter spontaner Magnetisierung und spontaner
Symmetriebrechung?
\end{myfrag} 
\subsection{102}
\begin{myfrag}
Leite die exakte Zustandsumme des Ising Modells ohne Magnetfeld in einer
Dimension her.
Beschreiben Sie wie man die Zustandssumme mit
Magnetfeld bestimmen kann. Wie heisst die Methode?
\end{myfrag} 
\subsection{103}
\begin{myfrag}
Beschreibe Sie das Konzept der Renormierungsgruppe allgemein.
Veranschauliche dieses Konzept an Hand des Ising Modells durch explizite
Rechnungen.
\end{myfrag} 