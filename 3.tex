\section{Mikrokanonisches ensemble}
\subsection{30}
\begin{myfrag}
Was ist die Definition der Entropie von einer allgemeinen Wahrscheinlichkeitsverteilung?
\end{myfrag}
\begin{equation}
	S=P(n)\logn P(n)
\end{equation}
\subsection{31}
\begin{myfrag}
Beschreibe das Konzept des Mikrokanonischen Ensembles. Was ist die
Wahrscheinlichkeitsverteilung im Mikrokanonischen Ensemble?
\end{myfrag}
Im Mikrokanonischen Ensemble wird angenommen, dass die Gesamtenergie und die Teilchenzahl erhalten sind, d.h. das System ist vollständig isoliert. Für die Verteilung von $n$ Energiequanten auf $N$ Oszillatoren oder ein vergleichbares Problem gilt
\begin{equation}
	Z_m=\Omega=\binkoef{N+n-1}{n}=\frac{(N+n-1)!}{(N-1)!n!}
\end{equation}
\subsection{32}
\begin{myfrag}
Mit welchem Ausdruck kann die Gesamtzahl der Zustände $\Omega$ im
Mikrokanonischen Ensemble berechnet werden? Was ist die Entropie?
\end{myfrag}
Siehe Frage 31 %TODO was zum geier???
\subsection{33}
\begin{myfrag}
Wann sind zwei Systeme im energetischen Gleichgewicht im Mikrokanonischen
Ensemble? Was bedeutet das für die Entropie?
\end{myfrag}
Die Systeme müssen die gleiche Energie haben. Die Entropie bleibt gleich. %TODO stimmt das?
\subsection{34}
\begin{myfrag}
Definiere Temperatur im Mikrokanonischen Ensemble. Wie werden
generalisierte Kräfte berechnet?
\end{myfrag}
Die Temperatur ist
\begin{equation}
	\frac{1}{T}=\partddd SEV \Rightarrow T=\left(\partddd SEV\right)^{-1}.
\end{equation}
Allgemein berechnet man die generalisierte Kraft $f_\alpha$ zum Potential $\alpha$ als
\begin{equation}
	f_\alpha=\partdd S\alpha .
\end{equation}
\subsection{35}
\begin{myfrag}
Was besagen die Stabilitätsbedingungen?
\end{myfrag}
Ein System $(1,2)$ ist stabil, wenn jede Änderung der Zustandsgrößen in den Teilsystemen zu einer Erniedrigung der Entropie führt, d.h.
\begin{align}
	S_{max}&=S_1(E_1,V_1)+S_2(E_2,V_2)\\
	\delta S&=S_1(E_1+\delta E_1,V_1+\delta V_1)+S_2(E_2+\delta E_2,V_2+\delta V_2)\le 0
\end{align}
\subsection{36}
\begin{myfrag}
Leite das Ideale Gasgesetz mit Hilfe des Mikrokanonischen Ensembles her.
Zeige, dass die durchschnittliche Energie pro Teilchen E=3kT/2 ist.
\end{myfrag}
Offensichtlich ist die Einteilchen-Zustandssumme $\Omega_1$ proportional zum Volumen $V$, also gilt $\Omega\propto V^N$.Durch Logarithmieren erhält man
\begin{equation}
	S=k_B\,\logn\Omega=k_BN\,\logn V+\const
\end{equation}
Außerdem gilt
\begin{alignat}{2}
	&p&&=T\partddd SV{N,E}=\frac{k_BTN}{V}\\
	\Leftrightarrow\qquad&pV&&=Nk_BT
\end{alignat}
\par Die Gesamtenergie erhält man, wenn man aus der Proportionalitätskonstante zwischen $V^N$ und $\Omega$ zusätzlich die Freiheitsgrade der Geschwindigkeit $v_i,\,i=x,y,z$ extrahiert. Wegen $v_x^2+v_y^2+v_z^2=\vec{v}^2=2mE$ für $N$ Teilchen erhält man
\begin{alignat}{2}
	&S&&=Nk_B\left(\logn V+\frac{3}{2}\logn E\right)+\const\\
	&\frac{1}{T}&&=\partddd SEV=\frac{\dfrac{3}{2}Nk_B}{E}\\
	\Leftrightarrow\qquad&E&&=\frac{3}{2}Nk_BT.
\end{alignat}
Damit ist die durchschnittliche Energie pro Teilchen $E=\frac{3}{2}k_BT$.
\subsection{37}
\begin{myfrag}
Betrachte ein vereinfachtes quantisiertes Modell für ein „Polymer“, wobei die
einzelnen Polymerglieder der Länge d nur nach links oder rechts zeigen können.
Berechne die generalisierte Kraft konjugiert zur Länge L mit Hilfe des
Mikrokanonischen Ensembles.
\end{myfrag}
%TODO
\subsection{38}
\begin{myfrag}
Was versteht man unter dem Loschmidt Paradoxon und dem Zermelo Paradoxon?
Beschreibe einen Maxwellschen Dämon.
\end{myfrag}
\begin{enumerate}[(i)]
\item\textit{Lohschmidt-Paradoxon:} Solange die Zeitumkehrsymmetrie nicht gebrochen ist, können spontane Zustandsänderungen auch rückwärts ablaufen, d.h. die Entropie kann dadurch spontan sinken. Dies stünde im Widerspruch zum zweiten Hauptsatz.
\item\textit{Zermelo-Paradoxon:} Endliche Systeme kömmen durch statistische Veränderungen stets in endlicher Zeit beliebig nahe an ihren Ausgangszustand zurück.
\item\textit{Maxwell-Dämon:} Blub.
\end{enumerate}
\section{Kanonisches Ensemble}
\subsection{39}
\begin{myfrag}
Beschreibe das Konzept des Kanonischen Ensembles und leite es mit Hilfe des
Mikrokanonischen Ensembles her. Was ist die Boltzmann Verteilung?
\end{myfrag}
Im Gegensatz zum mikrokanonischen Ensemle ist das kanonische Ensemble nicht vollständig isoliert. Die Teilchenzahl wird weiter konstant gehalten, allerdings ist Energieaustausch mit einem Wärmebad möglich, die Temperatur ist damit weiter konstant.\\
Wir schreiben zunächst die Energie als $E_\mathsf{ges}=\epsilon+E_\mathsf{Rest}$, wobei $\epsilon\ll E_\mathsf{Rest}$ die Energie im betrachteten System sei.