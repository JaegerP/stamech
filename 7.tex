\subsection{70}
\begin{myfrag}
Wie können p, N, E und S mit Hilfe des großkanonischen Potentials $\Phi$
berechnet werden?
\end{myfrag}
$N$ und $E$ werden als Erwartungwerte bestimmt, d.h.
\begin{align}
	N&=\sum\limits_{N=0}^\infty\sum\limits_{E(N)}\frac{Ne^{-\beta E}e^{-\alpha N}}{\mathscr{Z}}=-\frac{\partddd\alpha{\mathscr{Z}}\beta}{\mathscr{Z}}\nonumber\\
	&=\partddd{\logn \mathscr{Z}}\alpha\beta=k_BT\partddd{\logn \mathscr{Z}}\mu\beta=-\partddd\Phi\mu\beta\text{ bzw.}\\
	E&=\sum\limits_{N=0}^\infty\sum\limits_{E(N)}\frac{Ee^{-\beta E}e^{-\alpha N}}{\mathscr{Z}}=\partddd{\logn \mathscr{Z}}\beta\alpha.
\end{align}
$p$ und $S$ berechnet man wie üblich als Ableitung. Durch analoge Rechnung erhält man
\begin{align}
	p&=-\partddd EVT=-\partddd \Phi V{T,\mu}\text{ und}\\
	S&=-\partddd ETV=-\partddd \Phi T{V,\mu}.
\end{align}
\subsection{71}
\begin{myfrag}
Wende das Konzept des Großkanonischen Ensembles auf ein ideales
klassisches Gas an. Was ist z? Berechne E und p als Funktion von T, V und
N.
\end{myfrag}
\subsection{72}
\begin{myfrag}
Leite allgemeine Ausdrücke für die nichtwechselwirkenden bosonischen und
fermionischen großkanonischen Zustandssummen als Produkt über Einteilchenzustände
her. Zeige, dass die Energie und die Teilchenzahl als
Summe über Einteilchenzustände ausgedrückt werden können. Was versteht
man dementsprechend unter der Bose-Einstein und der Fermi-Dirac
Verteilung?
\end{myfrag}
Zunächst betrachtet man allgemein nicht wechselwirkende Quantenteilchen, das bedeutet, dass $N=\sum\limits_rn_r$ und $E=\sum\limits_r\epsilon_rn_r$ gelten. dabei sind die \ket r die Eigenzustände des Einteilchensystems in der Basis der Besetzungszahlen. Die Zustandssumme ist also
\begin{equation}
	\mathscr Z=\sum\limits_{N=0}^\infty\sum\limits_{\lbrace n_r\rbrace}e^{-\beta\sum\limits_rn_r\epsilon_r}e^{-\alpha\sum\limits_rn_r}.
\end{equation}
Die erste Summe kann dabei wegfallen, da $N=\sum\limits_rn_r$ gilt. Die Summen in den $e$-Funktionen kann man als Produkte schreiben und diese nach dem Distributivgesetz mit der verbleibenden Summe vertausen. Man erhält dann
\begin{align}
	\mathscr Z&=\sum\limits_{\lbrace n_r\rbrace}e^{-\beta\sum\limits_rn_r\epsilon_r}e^{-\alpha\sum\limits_rn_r}=\sum\limits_{\lbrace n_r\rbrace}\prod\limits_re^{-\beta n_r\epsilon_r}e^{-\alpha n_r}\nonumber\\
	&=\prod\limits_r\left(\sum\limits_{n_r=0}^\infty e^{-\beta n_r\epsilon_r}e^{-\alpha n_r}\right).
\end{align}
Ab hier muss unterschieden werden, ob Bosonen oder Fermionen berachtet werden, da für Fermionen nur die Besetzungszahlen $n_r=0,1$ möglich sind:
\begin{itemize}
	\item[\textbf{Bosonen:}]{
	\begin{alignat}{2}
		&\mathscr Z_B&&=\prod\limits_r\left(\sum\limits_{n_r=0}^\infty e^{-\beta n_r\epsilon_r}e^{-\alpha n_r}\right)\nonumber\\
		&&&=\prod\limits_r\frac{1}{1-e^{-\beta\epsilon_r}e^{-\alpha}}=\prod\limits_r\frac{1}{1-ze^{-\beta\epsilon_r}}\\
		\Rightarrow\qquad\qquad&\Phi_B&&=+k_BT\logn\mathscr Z_B=+k_BT\sum\limits_r\left(1-ze^{-\beta\epsilon_r}\right)
	\end{alignat}}
	\item[\textbf{Fermionen:}]{
	\begin{alignat}{2}
		&\mathscr Z_F&&=\prod\limits_r\left(\sum\limits_{n_r=0}^1 e^{-\beta n_r\epsilon_r}e^{-\alpha n_r}\right)\nonumber\\
		&&&=\prod\limits_r\left(1+e^{-\beta\epsilon_r}e^{-\alpha}\right)=\prod\limits_r\left(1+ze^{-\beta\epsilon_r}\right)\\
		\Rightarrow\qquad\qquad&\Phi_F&&=-k_BT\logn\mathscr Z_F=-k_BT\sum\limits_r\left(1+ze^{-\beta\epsilon_r}\right)
	\end{alignat}}
\end{itemize}

\subsection{73}
\begin{myfrag}
Was ist die Entwicklung in z für die Bose-Einstein und die Fermi-Dirac
Verteilung?
\end{myfrag}
Für $z\ll 1$ ist mit $\logn(1+\epsilon)\approx\epsilon$ $\Phi_B\approx\Phi_F\approx-k_BTz\sum\limits_re^{-\beta\epsilon_r}\approx\Phi_\mathsf{klassisch}$.
\subsection{74}
\begin{myfrag}
Leite die Einteilchen-Zustandsdichte $g(\omega )$ für ein ideales Quantengas in drei
Dimensionen her.
\end{myfrag}
\begin{align}
	g(\epsilon)&=\sum\limits_k\delta\left(\epsilon-\epsilon_k\right)\nonumber\\
	&\approx\int \frac{\dif^3k}{\Delta k^3}\delta\left(\epsilon-\frac{\hbar^2k^2}{2m}\right)\qquad\text{ mit }\Delta k^3=\frac{8\pi^3}{L^3}\\
	&=4\pi\frac{V}{2\pi^2}\int k^2\dif k\delta\left(\epsilon-\frac{\hbar^2k^2}{2m}\right)\nonumber\\
	&\overset{x=\frac{\hbar^2k^2}{2m}}{=}\frac{V}{2\pi^2}\int\frac{m}{\hbar^2}\dif x\sqrt{\frac{2mx}{\hbar^2}}\delta(\epsilon-x)\\
	&=\begin{dcases}\frac{V}{2\pi^2}\left(\frac{m}{\hbar}\right)^{\sfrac{3}{2}}\sqrt{\epsilon}\qquad&\text{für }\epsilon>0\\
	0\qquad&\text{für }\epsilon<0\end{dcases}
\end{align}

\subsection{75}
\begin{myfrag}
Finde Integralausdrücke für N und E als Funktion von z und T in einem
dreidimensionalen bosonischen Quantengas. Drücke die Integrale mit Hilfe
der polylogarithmischen Funktion $g_\nu (z)$  aus. Wie kann mit Hilfe dieser
Ausdrücke E als Funktion von N, V und T bestimmt werden (Eine Skizze ist
hilfreich hier).
\end{myfrag}
\subsection{76}
\begin{myfrag}
Zeige für ein dreidimensionales Quantengas, wie der Druck mit dem
Energieerwartungswert zusammenhängt.
\end{myfrag}
\section{Bose-Gas und BEC}
\subsection{77}
\begin{myfrag}
Erkläre das Konzept der Virialentwicklung am Beispiel eines idealen
Bosonen Gases. Berechne den ersten Koeffizienten.
\end{myfrag}
\subsection{78}
\begin{myfrag}
Argumentiere dass in der Berechnung der Teilchenzahl der Grundzustand
eines Bosonen Gases ab einer kritischen Temperatur gesondert behandelt
werden muss. Was ist die kritische Temperatur  $T_C$ ?
\end{myfrag}
\subsection{79}
\begin{myfrag}
Berechne den Kondensatsanteil x in einem dreidimensionalen Bosonengas als
Funktion der Temperatur und der Kondensationstemperatur $T_C$.
\end{myfrag}