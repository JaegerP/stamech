\subsection{70}
\begin{myfrag}
Wie können p, N, E und S mit Hilfe des großkanonischen Potentials $\Phi$
berechnet werden?
\end{myfrag}
$N$ und $E$ werden als Erwartungwerte bestimmt, d.h.
\begin{align}
	N&=\sum\limits_{N=0}^\infty\sum\limits_{E(N)}\frac{Ne^{-\beta E}e^{-\alpha N}}{\mathscr{Z}}=-\frac{\partddd\alpha{\mathscr{Z}}\beta}{\mathscr{Z}}\nonumber\\
	&=\partddd{\logn \mathscr{Z}}\alpha\beta=k_BT\partddd{\logn \mathscr{Z}}\mu\beta=-\partddd\Phi\mu\beta\text{ bzw.}\\
	E&=\sum\limits_{N=0}^\infty\sum\limits_{E(N)}\frac{Ee^{-\beta E}e^{-\alpha N}}{\mathscr{Z}}=\partddd{\logn \mathscr{Z}}\beta\alpha.
\end{align}
$p$ und $S$ berechnet man wie üblich als Ableitung
\subsection{71}
\begin{myfrag}
Wende das Konzept des Großkanonischen Ensembles auf ein ideales
klassisches Gas an. Was ist z? Berechne E und p als Funktion von T, V und
N.
\end{myfrag}
\subsection{72}
\begin{myfrag}
Leite allgemeine Ausdrücke für die nichtwechselwirkenden bosonischen und
fermionischen großkanonischen Zustandssummen als Produkt über Einteilchenzustände
her. Zeige, dass die Energie und die Teilchenzahl als
Summe über Einteilchenzustände ausgedrückt werden können. Was versteht
man dementsprechend unter der Bose-Einstein und der Fermi-Dirac
Verteilung?
\end{myfrag}
\subsection{73}
\begin{myfrag}
Was ist die Entwicklung in z für die Bose-Einstein und die Fermi-Dirac
Verteilung?
\end{myfrag}
\subsection{74}
\begin{myfrag}
Leite die Einteilchen-Zustandsdichte $g(\omega )$ für ein ideales Quantengas in drei
Dimensionen her.
\end{myfrag}
\subsection{75}
\begin{myfrag}
Finde Integralausdrücke für N und E als Funktion von z und T in einem
dreidimensionalen bosonischen Quantengas. Drücke die Integrale mit Hilfe
der polylogarithmischen Funktion $g_\nu (z)$  aus. Wie kann mit Hilfe dieser
Ausdrücke E als Funktion von N, V und T bestimmt werden (Eine Skizze ist
hilfreich hier).
\end{myfrag}
\subsection{76}
\begin{myfrag}
Zeige für ein dreidimensionales Quantengas, wie der Druck mit dem
Energieerwartungswert zusammenhängt.
\end{myfrag}
\section{Bose-Gas und BEC}
\subsection{77}
\begin{myfrag}
Erkläre das Konzept der Virialentwicklung am Beispiel eines idealen
Bosonen Gases. Berechne den ersten Koeffizienten.
\end{myfrag}
\subsection{78}
\begin{myfrag}
Argumentiere dass in der Berechnung der Teilchenzahl der Grundzustand
eines Bosonen Gases ab einer kritischen Temperatur gesondert behandelt
werden muss. Was ist die kritische Temperatur  $T_C$ ?
\end{myfrag}
\subsection{79}
\begin{myfrag}
Berechne den Kondensatsanteil x in einem dreidimensionalen Bosonengas als
Funktion der Temperatur und der Kondensationstemperatur $T_C$.
\end{myfrag}