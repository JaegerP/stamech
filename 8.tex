\subsection{80}
\begin{myfrag}
Berechne die Energie und die spezifische Wärme in einem idealen Bose-Gas
als Funktion der Temperatur bei gegebener Dichte (d.h. gegebener kritische
Temperatur Tc). Zeichne den ungefähren Verlauf der spezifischen Wärme CV
als Funktion von T eines bosonischen Gases.
\end{myfrag}
\subsection{81}
\begin{myfrag}
Argumentiere, dass in einem Bosegas für T<Tc der Druck proportional zu $T^{\dfrac{5}{2}}$
ist.
\end{myfrag}
\subsection{82}
\begin{myfrag}
Wie ändert sich das großkanonische Ensemble falls innere Freiheitsgrade
(z.B. Spin) berücksichtigt werden müssen? Was ist der Effekt für die
Quantenstatistik und Virialentwicklung falls es entartete innere Freiheitsgrade
gibt?
\end{myfrag}
\subsection{83}
\begin{myfrag}
Wie kann man bei einem bosonischen Gas Wechselwirkungseffekte durch
einen Molekularfeldansatz (MFT) berücksichtigen?
\end{myfrag}
\subsection{84}
\begin{myfrag}
Wende das großkanonische Ensemble auf ein bosonisches Photonengas mit
Entartung g=2 und $Ek=\hbar kc$ an. Berechne die Erwartungswerte der Energie,
der Teilchenzahl und des Druckes (mit Hilfe einer Integraltabelle). Warum
gibt es keine Bose-Einstein Kondensation in diesem Fall?
\end{myfrag}
\subsection{85}
\begin{myfrag}
Was sind die Ausdrücke für die Teilchenzahl und die Energie eines
dreidimensionalen fermionischen Gases als Funktion der Fugazität?
\end{myfrag}
\subsection{86}
\begin{myfrag}
Was ist die Virialentwicklung für ein ideales Fermionengas bis zur 1.
Ordnung?
\end{myfrag}
\subsection{87}
\begin{myfrag}
Wie ist die Fermi Energie $\epsilon _F$ definiert? Berechne den Druck und die Energie
für ein ideales dreidimensionales Fermigas bei T=0 als Funktion von N, V
und $\epsilon _F$.
\end{myfrag}
\subsection{88}
\begin{myfrag}
Was ist die Sommerfeld Entwicklung? Leite sie her. Stelle die Koeffizienten als
dimensionslose Integrale dar.
\end{myfrag}
\subsection{89}
\begin{myfrag}
Wende die Sommerfeld Entwicklung auf ein ideales Fermionen Gas in drei Dimensionen
an, um die Korrekturen für kleine Temperaturen zum chemischen Potential und zur
spezifischen Wärme zu bestimmen.
\end{myfrag}