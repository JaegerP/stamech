\subsection{90}
\begin{myfrag}
Wie hängt der Entwicklungsparameter $k_B T/\epsilon_F$ in der Sommerfeld Entwicklung mit dem
Parameter $\lambda^3 N/gV$ 3 N/gV zusammen (für ein ideales Fermigas)?
\end{myfrag}
\subsection{91}
\begin{myfrag}
Was besagt die Boltzmann Transport Gleichung und der Stoßzahlansatz? Erläutere die
Herleitung.
\end{myfrag}
\subsection{92}
\begin{myfrag}
Erläutere wie man einen Phasenübergang zwischen einer geordneten und einer
ungeordneten Phase quantitativ verstehen kann indem man die Freie Energie minimiert.
Was bedeutet dies für die Energie und Entropie bei hohen bzw. tiefen Temperaturen?
\end{myfrag}
\subsection{93}
\begin{myfrag}
Was ist die Ehrenfest Klassifikation? Was zeichnet einen Phasenübergang von 1.
Ordnung aus? Was versteht man unter einem Ordnungsparameter?
\end{myfrag}
\subsection{94}
\begin{myfrag}
Was besagt die Gibbssche Phasenregel für die Koexistenz verschiedener Phasen?
\end{myfrag}
\subsection{95}
\begin{myfrag}
Was ist die Clausius Clapeyron Gleichung? Leite sie her.
\end{myfrag} 
\subsection{96}
\begin{myfrag}
Skizziere das Phasendiagramm von Wasser. Leite Gleichungen für den
ungefähren Verlauf der Phasengrenzkurven zwischen Wasser/Eis, und
Wasser/Dampf her. Was ist ein Tripelpunkt? Welche Beziehung zwischen
den Steigungen der Phasengrenzkurven gibt es dort?
\end{myfrag} 
\subsection{97}
\begin{myfrag}
Wie lautet die Van-der-Waals Zustandsgleichung? Leite sie aus dem
Zustandsintegral her unter der Annahme, dass das Gas ungeordnet ist und
dass das Wechselwirkungspotential einen geeignet vereinfachten anziehenden
und abstoßenden Teil hat.
\end{myfrag} 
\subsection{98}
\begin{myfrag}
Was passiert quantitativ an einem kritischen Punkt? Wie sind die kritischen
Exponenten $\alpha, \beta , \gamma , \delta $ definiert? 
\end{myfrag} 
\subsection{99}
\begin{myfrag}
Was passiert wenn es einen Bereich gibt, in dem die berechnete Freie Energie
als Funktion von V konkav wird?
\end{myfrag} 