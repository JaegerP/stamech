%Fragenkatalog
\documentclass[12pt,a4paper,notitlepage]{report}

%\usepackage[T1]{fontenc}
\usepackage[utf8]{inputenc}
\usepackage[ngerman]{babel}
\usepackage{amsmath,xfrac,mathtools}
\usepackage{amsfonts,mathrsfs}
\usepackage{amssymb}
\usepackage{euscript,xspace}
\usepackage{graphicx}
\usepackage{geometry}
\usepackage{amsthm}
\usepackage{float,sidecap}
\usepackage{fancyhdr}
\usepackage{cancel}
\usepackage{paralist}
\usepackage{wrapfig} %hinzugefügt
\usepackage{relsize}
\setcounter{tocdepth}{1}
\newcommand{\theauthor}{Maximilian Kiefer-Emmanouilidis, Philipp Jaeger}
\author{\theauthor}
\date{}

\newcommand{\thetitle}{Fragenkatalog zu Thermodynamik und Statistik ohne Antworten\\\smaller{Prof. Eggert, Sommersemester 2014}}
\title{\textbf{\thetitle}\vspace{25mm}}

\geometry{a4paper,left=25mm,right=25mm, top=15mm, bottom=55mm}
%\setlength{\mathindent}{15mm}

\fancyhead{}
\fancyhead[R]{\textbf{\thetitle}\\\theauthor}
\fancyfoot[C]{\thepage}
\renewcommand{\headrulewidth}{0.4pt}
\renewcommand{\footrulewidth}{0pt}
\addtolength{\headheight}{70.5pt}
\pagestyle{fancy}

\renewcommand{\familydefault}{\sfdefault}
%hätte gerne nen sf-font

\newcommand{\betrag}[1]{\ensuremath {\mid #1 \mid}}
\newcommand{\dif}{\mathsf{d}}
\newcommand{\logn}{\mathsf{ln}}
\newcommand{\partd}[1]{\ensuremath {\frac{\partial}{\partial #1}}}
\newcommand{\partdd}[2]{\ensuremath {\left(\frac{\partial #1}{\partial #2}}\right)}
\newcommand{\partddd}[3]{\ensuremath {\left(\frac{\partial #1}{\partial #2}}\right)_{#3}}
\newcommand{\pseq}{\mathrel{\phantom{=}}}
\newcommand{\binkoef}[2]{\left(\begin{array}{c}{#1}\\{#2}\end{array}\right)}
\newcommand{\const}{\mathsf{const.}}
\newcommand{\ket}[1]{\ensuremath {\mid #1 \rangle}}
\newcommand{\bra}[1]{\ensuremath {\langle #1 \mid}}

\let\oldoverset\overset
\renewcommand{\overset}[2]{\oldoverset{\mathclap{#1}}{#2}\quad}

\newtheorem{mydef}{Definition}
%\newtheorem{myfrag}{}%Frage} doppelt nur die Nummerierung
\newenvironment{myfrag}{\begin{it}}{\end{it}\vspace{3mm}\par}
\newtheorem{mybem}{Bemerkung}
\newtheorem{mybei}{Beispiel}
\newtheorem{mybeh}{Behauptung}
\newtheorem{mybew}{Beweis}
\newtheorem{mycod}{Code}%sollte man als environment mit monospace bauen

\renewcommand{\thesubsection}{Frage}
\numberwithin{equation}{section}
\begin{document}
\maketitle
%\tableofcontents
\chapter{Thermodynamik}
\section{Thermodynamik und Allgeimeine Definitionen}
\subsection{1}
\begin{myfrag}
Was ist ein Makrozustand und ein Mikrozustand?
Was sind Zustandsgrößen?
\end{myfrag}
\subsection{2}
\begin{myfrag}
Was ist ein vollständiges Differential? Was ist ein integrierender Faktor?
In welchem Zusammenhang werden diese Konzepte in der
Thermodynamik benötigt (gebe Beispiele)?
\end{myfrag}
\subsection{3}
\begin{myfrag}
Was besagen der Nullte und der Erste Hauptsatz der Thermodynamik?
\end{myfrag}
\subsection{4}
\begin{myfrag}
Was ist eine generalisierte Kraft? Gebe ein Beispiel!
\end{myfrag}
\subsection{5}
\begin{myfrag}
Wie ist die Entropie in der Thermodynamik definiert? Was sind
reversible Prozesse?
\end{myfrag}
\subsection{6}
\begin{myfrag}
Was besagt der Zweite Hauptsatz der Thermodynamik?
\end{myfrag}
\subsection{7}
\begin{myfrag}
Gebe die thermodynamischen Definitionen für die Freie Energie F, die
Enthalpie H, die Freie Enthalpie G und das Grosskanonische Potential $\Phi $
an. Was sind jeweils die entsprechenden Differentiale? Wie können
thermodynamische Größen und generalisierte Kräfte mit den
thermodynamischen Potentialen berechnet werden?
\end{myfrag}
\subsection{8}
\begin{myfrag}
Was ist eine Legendre-Transformation?
\end{myfrag}
\subsection{9}
\begin{myfrag}
Was sind intensive und extensive thermodynamische Variablen? Gebe
Beispiele
\end{myfrag}
\subsection{10}
\begin{myfrag}
Was ist die Gibbs-Duhem Relation? Leite sie her! Was besagt die Gibbs-
Duhem Relation für die Freie Enthalpie G und das Grosskanonische
Potential $\Phi$?
\end{myfrag}
\subsection{11}
\begin{myfrag}
Definiere die Wärmekapazitäten Cp und CV, die Kompressibilität $\kappa_T$, den
Ausdehnungskoeffizienten $\alpha$ und den Spannungskoeffizienten $\gamma$ als
partielle Ableitungen. Berechne $\alpha, \gamma$, Cp und CV, für ein ideales
klassisches Gas.
\end{myfrag}
\subsection{12}
\begin{myfrag}
Leite die Umkehrrelation und die zyklische Relation für partielle
Ableitungen her.
\end{myfrag}
\subsection{13}
\begin{myfrag}
Zeige dass der Druck und allgemeine generalisierte Kräfte mit Hilfe einer
Ableitung der Entropie berechnet werden können.
\end{myfrag}
\subsection{14}
\begin{myfrag}
Leite einen Ausdruck für den Spannungskoeffizienten $\gamma$ als Funktion des
Ausdehnungskoeffizienten $\alpha$ und der Kompressibilität $\kappa_T$ her.
\end{myfrag}
\subsection{15}
\begin{myfrag}
Leite die vier Maxwell Relationen her!
\end{myfrag}
\subsection{16}
\begin{myfrag}
Leite einen allgemeinen Ausdruck von physikalischen Parametern für die
Differenz zwischen den Wärmekapazitäten $C_p$ und $C_V$ her.
\end{myfrag}
\subsection{17}
\begin{myfrag}
Betrachte die Druckabhängigkeit p(V) für einen adiabatischen Prozess. Leite mit
Hilfe geeigneter Relationen einen allgemeinen Ausdruck für $\dfrac{\partial V}{\partial p}|_S$ her. Was ist die
Adiabatengleichung unter der Annahme, dass $\sigma = \dfrac{C_p\gamma}{C_V\alpha}$ konstant bleibt? Was ist $\sigma = \dfrac{C_p\gamma}{C_V\alpha}$ für ein ideales klassisches Gas?
\end{myfrag}
\subsection{18}
\begin{myfrag}
Was ist mit dem Begriff reversibler Kreisprozess gemeint? Was ist die praktische
Bedeutung?
\end{myfrag}
\subsection{19}
\begin{myfrag}
Wie sind die Wirkungsgrade für Motoren, Kühlmaschinen und Wärmepumpen
definiert?
\end{myfrag}
\subsection{20}
\begin{myfrag}
Beweise dass alle reversiblen Kreisprozesse den gleichen Wirkungsgrad haben
müssen. Wie kann man dies zu einer thermodynamischen Definition der
Temperatur nutzen?
\end{myfrag}
\subsection{21}
\begin{myfrag}
Beschreibe den Carnot Kreisprozess und skizziere die dazugehörigen p-V und S-T
Diagramme. Berechne den Wirkungsgrad für reversible Kreisprozesse
\end{myfrag}
\subsection{22}
\begin{myfrag}
Beschreibe den Stirling Kreisprozess und skizziere das dazugehörige p-V
Diagramm. Wie könnte man einen realistischen Stirling Motor konstruieren?
Beschreibe eine geeignete Kolbenanordnung und die einzelnen Schritte des
Motors.
\end{myfrag}
\subsection{23}
\begin{myfrag}
Was sind wichtige Gründe, dass reale Maschinen einen geringeren Wirkungsgrad
haben? Wie kann die Entropieproduktion errechnet werden?
\end{myfrag}
\subsection{24}
\begin{myfrag}
Argumentiere, dass für effiziente Prozesse die Temperaturunterschiede beim
Überführen von Wärme möglichst klein sein sollten.
\end{myfrag}


\chapter{Statistische Mechanik}
\subsection{25}
\begin{myfrag}
\end{myfrag}
\subsection{26}
\begin{myfrag}
\end{myfrag}
\subsection{27}
\begin{myfrag}

\end{myfrag}
\subsection{28}
\begin{myfrag}
Was ist eine Wahrscheinlichkeitsverteilung? Wie werden Erwartungswerte
allgemein berechnet?
\end{myfrag}
\subsection{29}
\begin{myfrag}
Was besagt der zentrale Grenzwertsatz?
\end{myfrag}
\subsection{30}
\begin{myfrag}
Was ist die Definition der Entropie von einer allgemeinen Wahrscheinlichkeitsverteilung?
\end{myfrag}
\subsection{31}
\begin{myfrag}
Beschreibe das Konzept des Mikrokanonischen Ensembles. Was ist die
Wahrscheinlichkeitsverteilung im Mikrokanonischen Ensemble?
\end{myfrag}
\subsection{32}
\begin{myfrag}
Mit welchem Ausdruck kann die Gesamtzahl der Zustände $\Omega$ im
Mikrokanonischen Ensemble berechnet werden? Was ist die Entropie?
\end{myfrag}
\subsection{33}
\begin{myfrag}
Wann sind zwei Systeme im energetischen Gleichgewicht im Mikrokanonischen
Ensemble? Was bedeutet das für die Entropie?
\end{myfrag}
\subsection{34}
\begin{myfrag}
Definiere Temperatur im Mikrokanonischen Ensemble. Wie werden
generalisierte Kräfte berechnet?
\end{myfrag}
\subsection{35}
\begin{myfrag}
Was besagen die Stabilitätsbedingungen?
\end{myfrag}
\subsection{36}
\begin{myfrag}
Leite das Ideale Gasgesetz mit Hilfe des Mikrokanonischen Ensembles her.
Zeige, dass die durchschnittliche Energie pro Teilchen E=3kT/2 ist.
\end{myfrag}
\subsection{37}
\begin{myfrag}
Betrachte ein vereinfachtes quantisiertes Modell für ein „Polymer“, wobei die
einzelnen Polymerglieder der Länge d nur nach links oder rechts zeigen können.
Berechne die generalisierte Kraft konjugiert zur Länge L mit Hilfe des
Mikrokanonischen Ensembles.
\end{myfrag}
\subsection{38}
\begin{myfrag}
Was versteht man unter dem Loschmidt Paradoxon und dem Zermelo Paradoxon?
Beschreibe einen Maxwellschen Dämon.
\end{myfrag}
\subsection{39}
\begin{myfrag}
Beschreibe das Konzept des Kanonischen Ensembles und leite es mit Hilfe des
Mikrokanonischen Ensembles her. Was ist die Boltzmann Verteilung?
\end{myfrag}
\subsection{40}
\begin{myfrag}
Was ist die kanonische Zustandssumme? Leite den Erwartungswert der Energie
als Ausdruck der Zustandssumme her.
\end{myfrag}
\subsection{41}
\begin{myfrag}
Wie ist die Freie Energie definiert? Wie können Energie, Entropie sowie
generalisierte Kräfte (Druck, etc.) als Funktion der Freien Energie berechnet
werden?
\end{myfrag}
\subsection{42}
\begin{myfrag}
Vergleiche die Konzepte des Kanonischen und des Mikrokanonischen Ensembles
in einer Liste: Was ist die jeweilige physikalische Situation? Was ist jeweils die
Zustandssumme und das zentrale thermodynamische Potential? Wie werden
thermodynamische Größen (gen. Kräfte, Druck, Temperatur, Energie und
Entropie) bestimmt? Was sind die Besetzungswahrscheinlichkeiten?
\end{myfrag}
\subsection{43}
\begin{myfrag}
Wende die Methoden des Kanonischen Ensembles auf das Ideale Klassische Gas
an. Rechne die Erwartungswerte für den Druck, die Energie und die Entropie aus.
\end{myfrag}
\subsection{44}
\begin{myfrag}
Betrachte ein vereinfachtes quantisiertes Modell für ein „Polymer“, wobei die
einzelnen Polymerglieder der Länge d nur nach links oder rechts zeigen können.
Berechne den Erwartungswert der Länge L mit Hilfe des Kanonischen Ensembles
als Funktion der Kraft (siehe auch Frage 37.).
\end{myfrag}
\subsection{45}
\begin{myfrag}
Berechne die Mischentropie für zwei ideale Gase. Erläutere ausführlich das
Gibbssche Paradoxon und dessen Lösung.
\end{myfrag}
\subsection{46}
\begin{myfrag}
Wann sind zwei Teilsysteme unabhängig? Was passiert mit der Zustandssumme,
den Wahrscheinlichkeiten und den Erwartungswerten in diesem Fall? Was
versteht man unter einer Einteilchenzustandssumme?
\end{myfrag}
\subsection{47}
\begin{myfrag}
Argumentiere, dass die Änderung des statistischen Erwartungswertes der Energie
in Erwartungswerte für die Änderung von Wärme und Arbeit aufgeteilt werden
kann. Leite einen Zusammenhang mit Änderungen der Entropie und der
generalisierten Kräfte her.
\end{myfrag}
\subsection{48}
\begin{myfrag}
Zeige allgemein, dass Energieschwankungen im kanonischen Ensemble von der
spezifischen Wärme und der Temperatur bestimmt werden können. Wie verhalten
sich die absoluten und relativen Energiefluktuationen als Funktion von N in einem
idealen Gas?
\end{myfrag}
\subsection{49}
\begin{myfrag}
Was ist der Virialsatz? Leite ihn her!
\end{myfrag}
\subsection{50}
\begin{myfrag}
Was ist das Äquipartitionstheorem? Leite es her!
\end{myfrag}
\subsection{51}
\begin{myfrag}
Leite die Verteilung der Geschwindigkeiten in eine Richtung vx und die
Maxwellsche Geschwindigkeitsverteilung für |v| her.
\end{myfrag}
\subsection{52}
\begin{myfrag}
Berechne die Erwartungswerte $\left\langle v_x \right\rangle , \left\langle |v| \right\rangle , \left\langle v^2 \right\rangle $ und die wahrscheinlichste
Geschwindigkeit max $v_{max}$ für die Maxwellsche Geschwindigkeitsverteilung. Wie
groß ist die Fluktuation der Geschwindigkeiten?
\end{myfrag}
\subsection{53}
\begin{myfrag}
Welche zwei Bedingungen müssen gegeben sein, damit eine klassische
Näherung von Quantenfreiheitsgraden sinnvoll ist?
\end{myfrag}
\subsection{54}
\begin{myfrag}
Wie ist die thermische Wellenlänge definiert? Vergleiche mit der De-Broglie
Wellenlänge typischer Geschwindigkeiten aus der Maxwellschen
Geschwindigkeitsverteilung. Welche Bedingung muss für die Dichte gelten,
damit Quanteninterferenzeffekte vernachlässigbar sind?
\end{myfrag}
\subsection{55}
\begin{myfrag}
Berechne die Zustandssumme, den Energieerwartungswert und die
spezifische Wärme für einen harmonischen Quantenoszillator.
\end{myfrag}
\subsection{56}
\begin{myfrag}
Wie ist die Einteilchenzustandsdichte $g(\omega )$ definiert? Leite die
Einteilchenzustandsdichte für den Fall von drei-dimensionalen
Wellenvektoren mit linearer Dispersionsrelation her.
\end{myfrag}
\subsection{57}
\begin{myfrag}
Was ist das Plancksche Strahlungsgesetz? Leite es her. Was ist das
Rayleigh-Jeans Gesetz für kleine Frequenzen?
\end{myfrag}
\subsection{58}
\begin{myfrag}
Beschreibe das Einstein Modell für die spezifische Wärme von Festkörpern.
Leite den entsprechenden Ausdruck für die spezifische Wärme als Funktion
der Temperatur her. Was ist der Dulong-Petit Grenzwert für die spezifische
Wärme?
\end{myfrag}
\subsection{58}
\begin{myfrag}
Erkläre im Detail das Debye Modell für die spezifische Wärme von
Festkörpern. Was sind die wichtigen Näherungen? Definiere die Debye
Wellenvektor, Frequenz und Temperatur. Was ist das Tief- bzw.
Hochtemperaturverhalten für die spezifische Wärme als Funktion der
Temperatur?
\end{myfrag}
\subsection{59}
\begin{myfrag}
Welche Eigenschaften haben Materialen mit hoher bzw. niedriger Debye
Temperatur. Warum?
\end{myfrag}
\subsection{60}
\begin{myfrag}
Schreibe einen Ausdruck für die Einteilchen-Zustandssumme über die
quantisierten kinetischen Freiheitsgrade eines Gases. Was ist die
Hochtemperaturentwicklung der Energie und der spezifischen Wärme?
Welche Bedingung muss für die Längenskalen gelten, damit die
Hochtemperaturentwicklung gerechtfertigt ist?
\end{myfrag}
\subsection{61}
\begin{myfrag}
Was ist das Verhalten der spezifischen Wärme für die Vibrationsfreiheitsgrade
eines Molekülgases?
\end{myfrag}
\subsection{62}
\begin{myfrag}
Was ist die Zustandssumme über die quantisierten Rotationsfreiheitsgrade
eines Moleküls aus zwei verschiedenen Atomen? Berechne die
Tieftemperatur-Entwicklungen der Zustandssumme, der Energie und der
spezifischen Wärme.
\end{myfrag}
\subsection{63}
\begin{myfrag}
Erkläre wie eine Hochtemperatur-Entwicklung der Rotationszustandssumme
gemacht werden kann (die MacLaurin Summen Formel sei gegeben). Erkläre,
warum cv(T) ein Maximum als Funktion der Temperatur durchläuft.
\end{myfrag}
\subsection{64}
\begin{myfrag}
Erkläre allgemein wie sich die Eigenschaften eines diskreten Spektrums
(Entartung, Energieabstände) im Temperaturverlauf von cv(T) widerspiegeln
(mit Beispiel).
\end{myfrag}
\subsection{65}
\begin{myfrag}
Was muss bei der Zustandssumme über die quantisierten
Rotationsfreiheitsgrade eines zwei-atomigen Moleküls beachtet werden, wenn
das Molkül aus identischen Atomen besteht? Wie sieht dann die
Zustandssumme für die Fälle aus, dass der Kernspin s ganz- oder halbzahlig
ist?
\end{myfrag}
\subsection{66}
\begin{myfrag}
Was versteht man unter Ortho- und Para-Wasserstoff? Wie kann das
Verhältnis von Ortho und Para-Wasserstoff im Gleichgewicht berechnet
werden? Gegen welche Werte strebt das Verhältnis für sehr große und für
sehr kleine Temperaturen?
\end{myfrag}
\subsection{67}
\begin{myfrag}
Erkläre die Darstellung von fermionischen und bosonischen Wellenfunktionen
mit Hilfe von Besetzungszahlen. Was versteht man unter
statistischer Abstoßung bzw. Anziehung?
\end{myfrag}
\subsection{68}
\begin{myfrag}
Mache eine vereinfachte Herleitung für die Bose-Einstein und die Fermi-
Dirac Verteilungen als Summe über Besetzungszahlen eines Zustandes für
den Fall dass es kein chemisches Potential gibt.
\end{myfrag}
\subsection{69}
\begin{myfrag}
Beschreibe das Konzept des Großkanonischen Ensembles. Definiere die
großkanonische Zustandssumme, das großkanonische Potential $\Phi$, die
Fugazität z und das chemische Potential $\mu$.
\end{myfrag}
\subsection{70}
\begin{myfrag}
Wie können p, N, E und S mit Hilfe des großkanonischen Potentials $\Phi$
berechnet werden?
\end{myfrag}
\subsection{71}
\begin{myfrag}
Wende das Konzept des Großkanonischen Ensembles auf ein ideales
klassisches Gas an. Was ist z? Berechne E und p als Funktion von T, V und
N.
\end{myfrag}
\subsection{72}
\begin{myfrag}
Leite allgemeine Ausdrücke für die nichtwechselwirkenden bosonischen und
fermionischen großkanonischen Zustandssummen als Produkt über Einteilchenzustände
her. Zeige, dass die Energie und die Teilchenzahl als
Summe über Einteilchenzustände ausgedrückt werden können. Was versteht
man dementsprechend unter der Bose-Einstein und der Fermi-Dirac
Verteilung?
\end{myfrag}
\subsection{73}
\begin{myfrag}
Was ist die Entwicklung in z für die Bose-Einstein und die Fermi-Dirac
Verteilung?
\end{myfrag}
\subsection{74}
\begin{myfrag}
Leite die Einteilchen-Zustandsdichte $g(\omega )$ für ein ideales Quantengas in drei
Dimensionen her.
\end{myfrag}
\subsection{75}
\begin{myfrag}
Finde Integralausdrücke für N und E als Funktion von z und T in einem
dreidimensionalen bosonischen Quantengas. Drücke die Integrale mit Hilfe
der polylogarithmischen Funktion $g_\nu (z)$ aus. Wie kann mit Hilfe dieser
Ausdrücke E als Funktion von N, V und T bestimmt werden (Eine Skizze ist
hilfreich hier).
\end{myfrag}
\subsection{76}
\begin{myfrag}
Zeige für ein dreidimensionales Quantengas, wie der Druck mit dem
Energieerwartungswert zusammenhängt.
\end{myfrag}
\subsection{77}
\begin{myfrag}
Erkläre das Konzept der Virialentwicklung am Beispiel eines idealen
Bosonen Gases. Berechne den ersten Koeffizienten.
\end{myfrag}
\subsection{78}
\begin{myfrag}
Argumentiere dass in der Berechnung der Teilchenzahl der Grundzustand
eines Bosonen Gases ab einer kritischen Temperatur gesondert behandelt
werden muss. Was ist die kritische Temperatur $T_C$ ?
\end{myfrag}
\subsection{79}
\begin{myfrag}
Berechne den Kondensatsanteil x in einem dreidimensionalen Bosonengas als
Funktion der Temperatur und der Kondensationstemperatur $T_C$.
\end{myfrag}
\subsection{80}
\begin{myfrag}
Berechne die Energie und die spezifische Wärme in einem idealen Bose-Gas
als Funktion der Temperatur bei gegebener Dichte (d.h. gegebener kritische
Temperatur Tc). Zeichne den ungefähren Verlauf der spezifischen Wärme CV
als Funktion von T eines bosonischen Gases.
\end{myfrag}
\subsection{81}
\begin{myfrag}
Argumentiere, dass in einem Bosegas für T<Tc der Druck proportional zu $T^{\dfrac{5}{2}}$
ist.
\end{myfrag}
\subsection{82}
\begin{myfrag}
Wie ändert sich das großkanonische Ensemble falls innere Freiheitsgrade
(z.B. Spin) berücksichtigt werden müssen? Was ist der Effekt für die
Quantenstatistik und Virialentwicklung falls es entartete innere Freiheitsgrade
gibt?
\end{myfrag}
\subsection{83}
\begin{myfrag}
Wie kann man bei einem bosonischen Gas Wechselwirkungseffekte durch
einen Molekularfeldansatz (MFT) berücksichtigen?
\end{myfrag}
\subsection{84}
\begin{myfrag}
Wende das großkanonische Ensemble auf ein bosonisches Photonengas mit
Entartung g=2 und $Ek=\hbar kc$ an. Berechne die Erwartungswerte der Energie,
der Teilchenzahl und des Druckes (mit Hilfe einer Integraltabelle). Warum
gibt es keine Bose-Einstein Kondensation in diesem Fall?
\end{myfrag}
\subsection{85}
\begin{myfrag}
Was sind die Ausdrücke für die Teilchenzahl und die Energie eines
dreidimensionalen fermionischen Gases als Funktion der Fugazität?
\end{myfrag}
\subsection{86}
\begin{myfrag}
Was ist die Virialentwicklung für ein ideales Fermionengas bis zur 1.
Ordnung?
\end{myfrag}
\subsection{87}
\begin{myfrag}
Wie ist die Fermi Energie $\epsilon _F$ definiert? Berechne den Druck und die Energie
für ein ideales dreidimensionales Fermigas bei T=0 als Funktion von N, V
und $\epsilon _F$.
\end{myfrag}
\subsection{88}
\begin{myfrag}
\end{myfrag}
\subsection{89}
\begin{myfrag}
Wende die Sommerfeld Entwicklung auf ein ideales Fermionen Gas in drei Dimensionen
an, um die Korrekturen für kleine Temperaturen zum chemischen Potential und zur
spezifischen Wärme zu bestimmen.
\end{myfrag}
\subsection{90}
\begin{myfrag}
Wie hängt der Entwicklungsparameter $k_B T/\epsilon_F$ in der Sommerfeld Entwicklung mit dem
Parameter $\lambda^3 N/gV$ 3 N/gV zusammen (für ein ideales Fermigas)?
\end{myfrag}
\subsection{91}
\begin{myfrag}
Was besagt die Boltzmann Transport Gleichung und der Stoßzahlansatz? Erläutere die
Herleitung.
\end{myfrag}
\subsection{92}
\begin{myfrag}
Erläutere wie man einen Phasenübergang zwischen einer geordneten und einer
ungeordneten Phase quantitativ verstehen kann indem man die Freie Energie minimiert.
Was bedeutet dies für die Energie und Entropie bei hohen bzw. tiefen Temperaturen?
\end{myfrag}
\subsection{93}
\begin{myfrag}
Was ist die Ehrenfest Klassifikation? Was zeichnet einen Phasenübergang von 1.
Ordnung aus? Was versteht man unter einem Ordnungsparameter?
\end{myfrag}
\subsection{94}
\begin{myfrag}
Was besagt die Gibbssche Phasenregel für die Koexistenz verschiedener Phasen?
\end{myfrag}
\subsection{95}
\begin{myfrag}
Was ist die Clausius Clapeyron Gleichung? Leite sie her.
\end{myfrag}
\subsection{96}
\begin{myfrag}
Skizziere das Phasendiagramm von Wasser. Leite Gleichungen für den
ungefähren Verlauf der Phasengrenzkurven zwischen Wasser/Eis, und
Wasser/Dampf her. Was ist ein Tripelpunkt? Welche Beziehung zwischen
den Steigungen der Phasengrenzkurven gibt es dort?
\end{myfrag}
\subsection{97}
\begin{myfrag}
Wie lautet die Van-der-Waals Zustandsgleichung? Leite sie aus dem
Zustandsintegral her unter der Annahme, dass das Gas ungeordnet ist und
dass das Wechselwirkungspotential einen geeignet vereinfachten anziehenden
und abstoßenden Teil hat.
\end{myfrag}
\subsection{98}
\begin{myfrag}
Was passiert quantitativ an einem kritischen Punkt? Wie sind die kritischen
Exponenten $\alpha, \beta , \gamma , \delta $ definiert?
\end{myfrag}
\subsection{99}
\begin{myfrag}
Was passiert wenn es einen Bereich gibt, in dem die berechnete Freie Energie
als Funktion von V konkav wird?
\end{myfrag}
\subsection{100}
\begin{myfrag}
Beschreibe das Gittergasmodell. Was ist der Zusammenhang mit dem Ising
Modell?
\end{myfrag}
\subsection{101}
\begin{myfrag}
Schätze die Freie Energie einer Domänwand im Ising Modell in ein und zwei
Dimensionen ab und diskutiere die Möglichkeit eines Phasenüberganges.
Was versteht man unter spontaner Magnetisierung und spontaner
Symmetriebrechung?
\end{myfrag}
\subsection{102}
\begin{myfrag}
Leite die exakte Zustandsumme des Ising Modells ohne Magnetfeld in einer
Dimension her.
Beschreiben Sie wie man die Zustandssumme mit
Magnetfeld bestimmen kann. Wie heisst die Methode?
\end{myfrag}
\subsection{103}
\begin{myfrag}
Beschreibe Sie das Konzept der Renormierungsgruppe allgemein.
Veranschauliche dieses Konzept an Hand des Ising Modells durch explizite
Rechnungen.
\end{myfrag}
\end{document}